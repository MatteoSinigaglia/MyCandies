\section{Presentazione}
Nella presentazione è stato scelto scelto un layout fluido utilizzando dimensioni relative e in percentuale.
Sono stati definiti 3 intervalli per schermi desktop molto grandi, schermi portatili e mobile, definendo un foglio di stile per ognuno.
Gli elementi rilevanti del desktop sono concentri nel header già descritto nella parte di presentazione.
\subsection{Mobile}
Per la presentazione del mobile si è deciso di spostare i pulsanti "Accedi" e "Carrello" in alto a destra, mantenendo solo il "logo" nella zona "reach" dei mobile. Per i pulsanti sono state utilizzate delle icone, familiari agli utenti, per risparmiare spazio. La barra di ricerca è situata in centro al header, sia in senso verticale che orizzontale. Il menù è nascosto e implementato tramite Javascript ove disponibile. Nei dispositivi con Javascript disattivato il menù è visibile sotto l'header orientato al centro. 
La modalità landscape e portrait prevedono la stessa presentazione, escluso per i dispositivi più grandi (tablet) che in modalità landscape permettono la visualizzazione del sito desktop.
\subsection{Stampa}
Nella stampa sono stati rimossi tutti gli elementi ritenuti superflui. Sono stati eliminati: 
\begin{itemize}
    \item link;
    \item bottoni;
    \item immagini, escluse quelle di contenuto fondamentale;
    \item colori;
    \item header;
    \item i form di login e in generale quelli di piccole dimensioni;
    \item bordi.
\end{itemize}
Si è deciso di includere il footer nella stampa eliminando le voci superflue. Sono stati tolti i link social e i link nei contatti, mantenendo il contatto telefonico e punti vendita, che potrebbero essere stampati come promemoria. 
Sono stati rimossi tutti gli elementi che avevano solo uno scopo presentazionale dando priorità al contenuto.