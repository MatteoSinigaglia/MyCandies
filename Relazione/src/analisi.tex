\section{Analisi degli attori principali e del target di utenza della pagina}

\subsection{Amministratore} L'amministratore è la figura preposta alla gestione del sito web. Possiede privilegi speciali, quali: inserimento e rimozione degli articoli acquistabili, aggiornamento delle scorte disponibili. L'amministratore ha a disposizione una dashboard esclusiva dalla quale può visualizzare, aggiungere, modificare e rimuovere i prodotti disponibili. Inoltre può modificare e rimuovere gli utenti registrati nel sito tramite l'apposita sezione in cui può visualizzare tutti i dati di questi ultimi. L'amministratore, una volta autenticatosi, può accedere alla dashboard dalla voce "Gestione" aggiunta al menù principale. L'amministratore ha possibilità di navigare nel sito come un utente normale.

\subsection{Audience} L'utenza target del sito non è circoscritta ad una sola categoria di persone, i prodotti da noi proposti sono infatti di possibile interesse per chiunque. L'utenza non viene discriminata in base al proprio tenore di vita poiché il costo dei prodotti è molto variabile. Pertanto, sebbene alcuni prodotti non risultino accessibili agli utenti meno abbienti, la fascia d'utenza a cui il sito è indirizzato sarà molto ampia, grazie soprattutto all'elevata disponibilità di prodotti low-cost.
Ciascun utente (non amministratore) è classificato in base al proprio stato rispetto al sistema. Egli può essere non registrato, non autenticato o autenticato. L'autenticazione implica una previa registrazione. In base alla propria condizione egli può accedere a diverse funzionalità. L'utente non autenticato o non registrato che prova ad accedere all'area personale viene automaticamente re-indirizzato alla pagina dedicata a login e registrazione.

\subsection{Attori}
\subsubsection{Utente non registrato} \`E l'utente che non possiede un account presso il sito. Egli può registrarsi presso il sito compilando l'apposito form condiviso di login e registrazione.
\subsubsection{Utente non autenticato} Tale utente possiede un account personale ma non è al momento autenticato. Tale operazione può essere svolta tramite l'apposito form condiviso di login e registrazione.
\subsubsection{Utente autenticato} L'utente ha eseguito l'accesso e può ora visualizzare la propria area personale, presente dopo l'accesso nel menù di navigazione. Potrà inoltre effettuare il logout tramite l'apposita voce nel menù attivata in seguito al login.




